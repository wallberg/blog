\documentclass[]{article}

\usepackage[top=0.5in, bottom=0.5in, left=0.5in, right=0.25in]{geometry}
\usepackage{mathtools}
\usepackage{color, colortbl}
\usepackage[utf8]{inputenc}
\usepackage{csquotes}
\usepackage[english]{babel}
\usepackage{biblatex}

\addbibresource{twelvefold_way.bib}

% Falling Factorial
\newcommand{\fallingfactorial}[1]{%
  ^{\mspace{1mu}\underline{#1\mspace{-2mu}}\mspace{2mu}}%
}

% Stirling number of the second kind
\newcommand{\stirlingii}{\genfrac{\{}{\}}{0pt}{}}

% Color definition
\definecolor{Gray}{gray}{0.9}

% Document
\begin{document}

\title{The Twelvefold Way}
\author{Ben Wallberg}
\maketitle

\begin{abstract}
In combinatorics, the twelvefold way is a systematic classification of 12 related enumerative problems concerning two finite sets, which include the classical problems of counting permutations, combinations, multisets, and partitions either of a set or of a number. This is my cheat sheet, compiled from Stanley \cite{stanley_enumerative_1997} by way of Knuth \cite{knuth_art_2011} and Wikipedia \cite{noauthor_twelvefold_nodate}, et al. I use this aid to help me understand and solve Project Euler \cite{noauthor_project_nodate} problems.  Project Euler is a series of challenging mathematical/computer programming problems that require more than just mathematical insights to solve. 
\end{abstract}

% The Table
\begin{tabular}{ | p{.6in} | p{0.9in} || p{1.7in} | p{1.7in} | p{1.7in} |  }

\hline

% Headings

\rowcolor{Gray} 
\textbf{\em{f}-class}
& 
& \textbf{Any $f$} 
&  \textbf{Injective $f$} 
& \textbf{Surjective $f$} 
\\

& \textbf{\em{balls per urn}} 
& \textbf{unrestricted} 
& \textbf{$\leq$ 1} 
& \textbf{$\geq$ 1} 
\\

\hline

% Row 1

\rowcolor{Gray} 
$f$
&
& $n$-sequence in $\mathbf{X}$
& $n$-permutation of $\mathbf{X}$
& composition of $\mathbf{N}$ with $x$ subsets
\\

& $n$ labeled balls, $m$ labeled urns
& $n$-tuples of $m$ things
& $n$-permutation of $m$ things
& partitions of $\{1,\ldots,n\}$ into $m$ ordered parts
\\

&
& $$x^n$$
& $$x\fallingfactorial{n}$$
& $$x! \stirlingii{n}{x}$$
\\

\hline

% Row 2

\rowcolor{Gray} 
$f \circ \mathbf{S}_n$
&
& $n$-multisubset of $\mathbf{X}$ 
& $n$-subset of $\mathbf{X}$ 
& composition of $n$ with $x$ terms
\\

& $n$ unlabeled balls, $m$ labeled urns
& $n$-multicombinations of $m$ things
& $n$-combinations of $m$ things
& compositions of $n$ into $m$ parts
\\

&
& $$\binom{x+n-1}{n}$$
& $$\binom{x}{n}$$
& $$\binom{n-1}{n-x}$$
\\

\hline

% Row 3

\rowcolor{Gray} 
$\mathbf{S}_x \circ f$
&
& partition of $\mathbf{N}$ into $\leq x$ subsets
& partition of $\mathbf{N}$ into $\leq x$ elements
& partition of $\mathbf{N}$ into $x$ subsets
\\

& $n$ labeled balls, $m$ unlabeled urns
& partitions of $\{1,\ldots,n\}$ into $\leq m$ parts
& $n$ pigeons into $m$ holes
& partitions of $\{1,\ldots,n\}$ into $m$ parts
\\

&
& $$\sum_{k=0}^{x} \stirlingii{n}{x}$$
& $$[n \leq x]$$
& $$\stirlingii{n}{x}$$
\\

\hline

%Row 4

\rowcolor{Gray} 
$\mathbf{S}_x \circ f \circ \mathbf{S}_n$
&
& partition of $n$ into $\leq x$ parts
& partition of $n$ into $\leq x$ parts 1
& partition of $n$ into $x$ parts
\\

& $n$ unlabeled balls, $m$ unlabeled urns
& partitions of $n$ into $\leq m$ parts
& $n$ pigeons into $m$ holes
& partitions of $n$ into $m$ parts
\\

& 
& $$p_x(n+x)$$
& $$[n \leq x]$$
& $$p_x(n)$$
\\

\hline
\end{tabular}

\printbibliography

\end{document}